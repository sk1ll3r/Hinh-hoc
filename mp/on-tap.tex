\chapter{Ôn tập một số kiến thức cơ bản trong chương trình PTTH}
\section{Các phép tính về vectơ}
    \subsection{Phép cộng vectơ}
        Cho 2 vectơ $\vecb u$ và $\vecb v$, đặt $\vec{OA} = \vecb u$, $\vec{OB} = \vecb v$. Gọi $C$ là đỉnh của hình bình hành $OACB$. Ký hiệu $\vecb w = \vec{OC}$ là tổng của $\vecb u$ và $\vecb v$.
        \begin{equation}
            \vecb w = \vecb u + \vecb v
        \end{equation}

        Chú ý: định nghĩa tổng vectơ nói trên không phụ thuộc vào việc chọn điểm đặt $O$. Tính chất của phép cộng vectơ:
        \begin{align}
            \vecb u + \vecb v               &= \vecb v + \vecb u                & \text{(Tính giao hoàn)} \\
            (\vecb u + \vecb v) + \vecb w   &= \vecb u + (\vecb v + \vecb w)    & \text{(Tính kết hợp)} \\
            \vecb u + \vecb 0               &= \vecb u \\
            \vecb u + (-\vecb u)            &= \vecb 0 
        \end{align}

    \subsection{Nhân vectơ với một số thực}
    \subsection{Tích vô hướng của hai vectơ}
    \subsection{Tích vectơ của hai vectơ trong không gian}
    \subsection{Tích hỗn hợp của 3 vectơ}

\section{Hệ tọa độ ĐềCác vuông góc  trong mặt phẳng và trong không gian}
\section{Đường thẳng}
\section{Tam giác và vòng tròn}
\section{Các phép biến  hình trong mặt phẳng}
