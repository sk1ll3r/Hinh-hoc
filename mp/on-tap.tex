\chapter{Ôn tập một số kiến thức cơ bản trong chương trình PTTH}
\section{Các phép tính về vectơ}
    \subsection{Phép cộng vectơ}
        Cho 2 vectơ $\vecb u$ và $\vecb v$, đặt $\vec{OA} = \vecb u$, $\vec{OB} = \vecb v$. Gọi $C$ là đỉnh của hình bình hành $OACB$. Ký hiệu $\vecb w = \vec{OC}$ là tổng của $\vecb u$ và $\vecb v$.
        \begin{equation}
            \vecb w = \vecb u + \vecb v
        \end{equation}

        Chú ý: định nghĩa tổng vectơ nói trên không phụ thuộc vào việc chọn điểm đặt $O$. Tính chất của phép cộng vectơ:
        \begin{align}
            \vecb u + \vecb v               &= \vecb v + \vecb u                & \text{(Tính giao hoàn)} \\
            (\vecb u + \vecb v) + \vecb w   &= \vecb u + (\vecb v + \vecb w)    & \text{(Tính kết hợp)} \\
            \vecb u + \vecb 0               &= \vecb u \\
            \vecb u + (-\vecb u)            &= \vecb 0 
        \end{align}

    \subsection{Nhân vectơ với một số thực}
        Tích của số $a$ với vectơ $\vecb u$ là một vectơ (kí hiệu là $a\vecb u$) cùng phương với $\vecb u$, cùng chiều với $\vecb u$ nếu $a > 0$, ngược chiều với $\vecb u$ nếu $a < 0$ có độ dài $= |a| |\vecb u|$. Ta nói $a \vecb u$ cộng tuyến với $\vecb u$. Tính chất:
        \begin{align}
            a(\vecb u + \vecb v)    &= a\vecb u + a\vecb v \\
            (a + b)\vecb u          &= a\vecb u + b\vecb v \\
            1\vecb u                &= \vecb u \\
            0\vecb u                &= \vecb 0
        \end{align}

    \subsection{Tích vô hướng của hai vectơ}
        Tích vô hướng của $\vecb u$ và $\vecb v$ (kí hiệu là $\vecb u \cdot \vecb v$) là một số $= |\vecb u||\vecb v|\cos{(\vecb u, \vecb v)}$; $(\vecb u, \vecb v)$ kí hiệu là số đo của góc tạo bởi $\vecb u$ và $\vecb v$.
        
        Suy ra
        \begin{align}
            \vecb u \cdot \vecb u                               &= \vecb v \cdot \vecb u    & \text{(tính giao hoàn)} \\
            \vecb u \cdot (\vecb v + \vecb w)                   &= \vecb u \cdot \vecb v + \vecb u \cdot \vecb w & \text{(tính phân phói)} \\
            \vecb u \cdot \vecb u \text{ (kí hiệu là } \vecb u^2 \text{)} &= \vecb u^2 & \text{Ta có } (\vecb u + \vecb v)^2 = \vecb u^2 + 2\vecb u \cdot \vecb v + \vecb v^2
        \end{align}

        Ta có
        \begin{align}
            \vecb u \cdot \vecb v   &= 0 \iff \vecb u \perp \vecb v & \text{(hoặc một trong hai vectơ đó} = \vecb 0 \text{)} \\
            \vecb u \cdot \vecb v &= |\vecb u||\vecb v| &\text{(nếu hai vectơ cùng chiều)} \\
            \vecb u \cdot \vecb v &= -|\vecb u||\vecb v| &\text{(nếu hai vectơ ngược chiều)} 
        \end{align}

    \subsection{Tích vectơ của hai vectơ trong không gian}
    \subsection{Tích hỗn hợp của 3 vectơ}

\section{Hệ tọa độ ĐềCác vuông góc  trong mặt phẳng và trong không gian}
\section{Đường thẳng}
\section{Tam giác và vòng tròn}
\section{Các phép biến  hình trong mặt phẳng}
